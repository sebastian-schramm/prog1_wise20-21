\subsection{Teilaufgabe 2}
\subsubsection{Aufgabenstellung}
Wir sollen ein Programm schreiben welches welches 2 nxn Matrizen miteinander Addieren und Multiplizieren kann.

\subsubsection{Anforderungsdefinition}
\begin{enumerate}
	\item Addiere zwei nxn Matrizen.
	\item Multipliziere zwei nxn Matrizen.
\end{enumerate}

\subsubsection{Entwurf}
\subsection{Teilaufgabe 2}
\subsubsection{Aufgabenstellung}
Wir sollen ein Programm schreiben welches welches 2 nxn Matrizen miteinander Addieren und Multiplizieren kann.

\subsubsection{Anforderungsdefinition}
\begin{enumerate}
	\item Addiere zwei nxn Matrizen.
	\item Multipliziere zwei nxn Matrizen.
\end{enumerate}

\subsubsection{Entwurf}
\subsection{Teilaufgabe 2}
\subsubsection{Aufgabenstellung}
Wir sollen ein Programm schreiben welches welches 2 nxn Matrizen miteinander Addieren und Multiplizieren kann.

\subsubsection{Anforderungsdefinition}
\begin{enumerate}
	\item Addiere zwei nxn Matrizen.
	\item Multipliziere zwei nxn Matrizen.
\end{enumerate}

\subsubsection{Entwurf}
\subsection{Teilaufgabe 2}
\subsubsection{Aufgabenstellung}
Wir sollen ein Programm schreiben welches welches 2 nxn Matrizen miteinander Addieren und Multiplizieren kann.

\subsubsection{Anforderungsdefinition}
\begin{enumerate}
	\item Addiere zwei nxn Matrizen.
	\item Multipliziere zwei nxn Matrizen.
\end{enumerate}

\subsubsection{Entwurf}
\input{uml/chapter_04_part2.tex}

\subsubsection{Quellcode}
\paragraph{Matrizen.java}\
\lstinputlisting[language = Java , frame = trBL , escapeinside={(*@}{@*)}]{../chapter_04/src/chapter_04/Matrizen.java}

\subsubsection{Testdokumentation}

\subsubsection{Benutzungshinweise}
Navigieren Sie in der Kommandozeile zum dem Ordner, wo sich die Java Datei befindet.
Danach führen sie "javac Matrizen.java\dq \space auf. Jetzt können Sie das Programm mit
"java Matrizen\dq \space starten. Nach dem das Programm gestartet ist, können Sie die
grö\ss e der Matrix angeben.

\subsubsection{Anwendungsbeispiel}
Nach dem Aufruf von java Matrizen, sollten wir nun folgendes sehen:
\begin{lstlisting}[frame = trBL , escapeinside={(*@}{@*)}]
[sebastian@laptop bin]$ java Matrizen
Matrix A:
70	50	
16	52	

Matrix B:
80	75	
11	33	

Addition von A und B:
150	125	
27	85	

Multiplikation von A und B:
For Schleife
6150	6900	
1852	2916
[sebastian@laptop bin]$  
\end{lstlisting}


\subsubsection{Quellcode}
\paragraph{Matrizen.java}\
\lstinputlisting[language = Java , frame = trBL , escapeinside={(*@}{@*)}]{../chapter_04/src/chapter_04/Matrizen.java}

\subsubsection{Testdokumentation}

\subsubsection{Benutzungshinweise}
Navigieren Sie in der Kommandozeile zum dem Ordner, wo sich die Java Datei befindet.
Danach führen sie "javac Matrizen.java\dq \space auf. Jetzt können Sie das Programm mit
"java Matrizen\dq \space starten. Nach dem das Programm gestartet ist, können Sie die
grö\ss e der Matrix angeben.

\subsubsection{Anwendungsbeispiel}
Nach dem Aufruf von java Matrizen, sollten wir nun folgendes sehen:
\begin{lstlisting}[frame = trBL , escapeinside={(*@}{@*)}]
[sebastian@laptop bin]$ java Matrizen
Matrix A:
70	50	
16	52	

Matrix B:
80	75	
11	33	

Addition von A und B:
150	125	
27	85	

Multiplikation von A und B:
For Schleife
6150	6900	
1852	2916
[sebastian@laptop bin]$  
\end{lstlisting}


\subsubsection{Quellcode}
\paragraph{Matrizen.java}\
\lstinputlisting[language = Java , frame = trBL , escapeinside={(*@}{@*)}]{../chapter_04/src/chapter_04/Matrizen.java}

\subsubsection{Testdokumentation}

\subsubsection{Benutzungshinweise}
Navigieren Sie in der Kommandozeile zum dem Ordner, wo sich die Java Datei befindet.
Danach führen sie "javac Matrizen.java\dq \space auf. Jetzt können Sie das Programm mit
"java Matrizen\dq \space starten. Nach dem das Programm gestartet ist, können Sie die
grö\ss e der Matrix angeben.

\subsubsection{Anwendungsbeispiel}
Nach dem Aufruf von java Matrizen, sollten wir nun folgendes sehen:
\begin{lstlisting}[frame = trBL , escapeinside={(*@}{@*)}]
[sebastian@laptop bin]$ java Matrizen
Matrix A:
70	50	
16	52	

Matrix B:
80	75	
11	33	

Addition von A und B:
150	125	
27	85	

Multiplikation von A und B:
For Schleife
6150	6900	
1852	2916
[sebastian@laptop bin]$  
\end{lstlisting}


\subsubsection{Quellcode}
\paragraph{Matrizen.java}\
\lstinputlisting[language = Java , frame = trBL , escapeinside={(*@}{@*)}]{../chapter_04/src/chapter_04/Matrizen.java}

\subsubsection{Testdokumentation}

\subsubsection{Benutzungshinweise}
Navigieren Sie in der Kommandozeile zum dem Ordner, wo sich die Java Datei befindet.
Danach führen sie "javac Matrizen.java\dq \space auf. Jetzt können Sie das Programm mit
"java Matrizen\dq \space starten. Nach dem das Programm gestartet ist, können Sie die
grö\ss e der Matrix angeben.

\subsubsection{Anwendungsbeispiel}
Nach dem Aufruf von java Matrizen, sollten wir nun folgendes sehen:
\begin{lstlisting}[frame = trBL , escapeinside={(*@}{@*)}]
[sebastian@laptop bin]$ java Matrizen
Matrix A:
70	50	
16	52	

Matrix B:
80	75	
11	33	

Addition von A und B:
150	125	
27	85	

Multiplikation von A und B:
For Schleife
6150	6900	
1852	2916
[sebastian@laptop bin]$  
\end{lstlisting}
