\subsection{Teilaufgabe 2}
\subsubsection{Aufgabenstellung}
Wir sollen ein Programm schreiben welches welches 2 nxn Matrizen miteinander Addieren und Multiplizieren kann.

\subsubsection{Anforderungsdefinition}
\begin{enumerate}
	\item Addiere zwei nxn Matrizen.
	\item Multipliziere zwei nxn Matrizen.
\end{enumerate}

\subsubsection{Entwurf}

\subsubsection{Quellcode}
\paragraph{Matrizen.java}\
\lstinputlisting[language = Java , frame = trBL , escapeinside={(*@}{@*)}]{../chapter_04/src/chapter_04/Matrizen.java}

\subsubsection{Testdokumentation}

\subsubsection{Benutzungshinweise}
Navigieren Sie in der Kommandozeile zum dem Ordner, wo sich die Java Datei befindet.
Danach führen sie "javac Matrizen.java\dq \space auf. Jetzt können Sie das Programm mit
"java Matrizen\dq \space starten. Nach dem das Programm gestartet ist, können Sie die
grö\ss e der Matrix angeben.

\subsubsection{Anwendungsbeispiel}
Nach dem Aufruf von java Matrizen, sollten wir nun folgendes sehen:
\begin{lstlisting}[frame = trBL , escapeinside={(*@}{@*)}]
[sebastian@laptop bin]$ java Matrizen 
Dieses Programm berechnet eine zufällig erstellte nxn Matrix
Geben sie n an: 5
Matrix A:
22	54	23	36	35	
49	98	67	82	23	
78	62	36	60	2	
54	2	74	48	95	
58	2	86	3	75	

Matrix B:
84	7	63	44	84	
69	55	76	83	97	
60	98	66	41	4	
27	6	71	96	24	
40	8	21	66	38	

Addition von A und B:
106	61	86	80	119	
118	153	143	165	120	
138	160	102	101	6	
81	8	145	144	119	
98	10	107	69	113	

Multiplikation von A und B:
9326	5874	10299	12159	9372	
18032	12975	21262	22427	16732	
14690	7860	16304	15946	14226	
14210	8788	13841	16454	9788	
13251	9562	11270	11482	8332
[sebastian@laptop bin]$  
\end{lstlisting}
