\section{Kapitel 5}
\subsection{Teilaufgabe 1}
\subsubsection{Aufgabenstellung}
In der ersten Teilaufgabe sollen wir ein Kleines simples Programm schreiben,
welches die Nebeneffekte in Java verdeutlicht.

\subsubsection{Anforderungsdefinition}
\begin{enumerate}
	\item Nebeneffekte verdeutlichen.
\end{enumerate}

\subsubsection{Entwurf}


\subsubsection{Quelltext}
\paragraph{Nebeneffekte.java}\
\lstinputlisting[language = Java , frame = trBL , escapeinside={(*@}{@*)}]{../chapter_05/src/chapter_05/Nebeneffekte.java}

\subsubsection{Testdokumentation}

\subsubsection{Benutzungshinweise}
Keine Besonderen Benutzungshinweise.
Das Programm muss lediglich nur ausgeführt werden.

\subsubsection{Anwendungsbeispiel}
Nach dem man das Programm gestartet hat, sollte folgende Ausgabe erscheinen:
\begin{lstlisting}[frame = trBL , escapeinside={(*@}{@*)}]
[sebastian@laptop bin]$ java Nebeneffekte 
Der Wert von x lautet: 10
Der Wert von y lautet: 23
Der Wert von z lautet: 32
[sebastian@laptop bin]$ 
\end{lstlisting}
