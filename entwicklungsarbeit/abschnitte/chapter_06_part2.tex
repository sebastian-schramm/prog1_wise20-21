\section{Kapitel 6}
\subsection{Teilaufgabe 2}
\subsubsection{Aufgabenstellung}
In der zweiten Teilaufgabe sollten wir Sprunganweisungen in Java Sinvoll verdeutlichen.

\subsubsection{Anforderungsdefinition}
\begin{enumerate}
	\item Verwenden sie Sprunganweisungen.
	\item Mindestens ein switch-Anweisung.
\end{enumerate}

\subsubsection{Entwurf}


\subsubsection{Quelltext}
\paragraph{Sprunganweisungen.java}\
\lstinputlisting[language = Java , frame = trBL , escapeinside={(*@}{@*)}]{../chapter_06/src/chapter_06/Sprunganweisungen.java}

\subsubsection{Testdokumentation}

\subsubsection{Benutzungshinweise}
Nach dem aufrufen des Programmes, wird der nutzer aufgefordert seine NutzerID anzugebe,j
sowie anschlie\ss end sein Passwort. Bei inkorrekter eingaben, wird man erneut aufgeforder
die Daten einzutippen.

\subsubsection{Anwendungsbeispiel}
Bei Erfolgreicher Anmeldung:
\begin{lstlisting}[frame = trBL , escapeinside={(*@}{@*)}]
[sebastian@laptop bin]$ java Sprunganweisungen 	
Wilkommen...!
ID      : 1
Passwort: hallo
Juhu, Sie haben sich eingeloggt
[sebastian@laptop bin]$ 
\end{lstlisting}
Bei inkorrekter Anmeldung:
\begin{lstlisting}[frame = trBL , escapeinside={(*@}{@*)}]
[sebastian@laptop bin]$ java Sprunganweisungen 	
Wilkommen...!
ID      : 12
Passwort: qwert
Ihre Angaben sind leider falsch, versuchen Sie es erneut.
Wilkommen...!
ID      : 
[sebastian@laptop bin]$ 
\end{lstlisting}