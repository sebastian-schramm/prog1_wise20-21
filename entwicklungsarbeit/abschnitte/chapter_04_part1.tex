\section{Kapitel 4}
\subsection{Teilaufgabe 1}
\subsubsection{Aufgabenstellung}
Wir sollen ein Programm schreiben welches Prüft ob zwei Referenzen gleich sind.

\subsubsection{Anforderungsdefinition}
\begin{enumerate}
	\item Prüfe ob zwei Referenzen gleich sind.
\end{enumerate}

\subsubsection{Entwurf}
\section{Kapitel 4}
\subsection{Teilaufgabe 1}
\subsubsection{Aufgabenstellung}
Wir sollen ein Programm schreiben welches Prüft ob zwei Referenzen gleich sind.

\subsubsection{Anforderungsdefinition}
\begin{enumerate}
	\item Prüfe ob zwei Referenzen gleich sind.
\end{enumerate}

\subsubsection{Entwurf}
\section{Kapitel 4}
\subsection{Teilaufgabe 1}
\subsubsection{Aufgabenstellung}
Wir sollen ein Programm schreiben welches Prüft ob zwei Referenzen gleich sind.

\subsubsection{Anforderungsdefinition}
\begin{enumerate}
	\item Prüfe ob zwei Referenzen gleich sind.
\end{enumerate}

\subsubsection{Entwurf}
\section{Kapitel 4}
\subsection{Teilaufgabe 1}
\subsubsection{Aufgabenstellung}
Wir sollen ein Programm schreiben welches Prüft ob zwei Referenzen gleich sind.

\subsubsection{Anforderungsdefinition}
\begin{enumerate}
	\item Prüfe ob zwei Referenzen gleich sind.
\end{enumerate}

\subsubsection{Entwurf}
\input{uml/chapter_04_part1.tex}

\subsubsection{Quellcode}
\paragraph{Referenzen.java}\
\lstinputlisting[language = Java , frame = trBL , escapeinside={(*@}{@*)}]{../chapter_04/src/chapter_04/Referenzen.java}
\paragraph{Punkt.java}\
\lstinputlisting[language = Java , frame = trBL , escapeinside={(*@}{@*)}]{../chapter_04/src/chapter_04/Punkt.java}

\subsubsection{Testdokumentation}

\subsubsection{Benutzungshinweise}
Navigieren Sie in der Kommandozeile zum dem Ordner, wo sich die Java Datei befindet.
Danach führen sie "javac Referenzen.java\dq \space auf. Jetzt können Sie das Programm mit
"java Referenzen\dq \space starten.

\subsubsection{Anwendungsbeispiel}
Nach dem Aufruf von java Referenzen, sollten wir nun folgendes sehen:
\begin{lstlisting}[frame = trBL , escapeinside={(*@}{@*)}]
[sebastian@laptop bin]$ java Referenzen 
Ist ungleich
Ist ungleich
Ist gleich
[sebastian@laptop bin]$  
\end{lstlisting}


\subsubsection{Quellcode}
\paragraph{Referenzen.java}\
\lstinputlisting[language = Java , frame = trBL , escapeinside={(*@}{@*)}]{../chapter_04/src/chapter_04/Referenzen.java}
\paragraph{Punkt.java}\
\lstinputlisting[language = Java , frame = trBL , escapeinside={(*@}{@*)}]{../chapter_04/src/chapter_04/Punkt.java}

\subsubsection{Testdokumentation}

\subsubsection{Benutzungshinweise}
Navigieren Sie in der Kommandozeile zum dem Ordner, wo sich die Java Datei befindet.
Danach führen sie "javac Referenzen.java\dq \space auf. Jetzt können Sie das Programm mit
"java Referenzen\dq \space starten.

\subsubsection{Anwendungsbeispiel}
Nach dem Aufruf von java Referenzen, sollten wir nun folgendes sehen:
\begin{lstlisting}[frame = trBL , escapeinside={(*@}{@*)}]
[sebastian@laptop bin]$ java Referenzen 
Ist ungleich
Ist ungleich
Ist gleich
[sebastian@laptop bin]$  
\end{lstlisting}


\subsubsection{Quellcode}
\paragraph{Referenzen.java}\
\lstinputlisting[language = Java , frame = trBL , escapeinside={(*@}{@*)}]{../chapter_04/src/chapter_04/Referenzen.java}
\paragraph{Punkt.java}\
\lstinputlisting[language = Java , frame = trBL , escapeinside={(*@}{@*)}]{../chapter_04/src/chapter_04/Punkt.java}

\subsubsection{Testdokumentation}

\subsubsection{Benutzungshinweise}
Navigieren Sie in der Kommandozeile zum dem Ordner, wo sich die Java Datei befindet.
Danach führen sie "javac Referenzen.java\dq \space auf. Jetzt können Sie das Programm mit
"java Referenzen\dq \space starten.

\subsubsection{Anwendungsbeispiel}
Nach dem Aufruf von java Referenzen, sollten wir nun folgendes sehen:
\begin{lstlisting}[frame = trBL , escapeinside={(*@}{@*)}]
[sebastian@laptop bin]$ java Referenzen 
Ist ungleich
Ist ungleich
Ist gleich
[sebastian@laptop bin]$  
\end{lstlisting}


\subsubsection{Quellcode}
\paragraph{Referenzen.java}\
\lstinputlisting[language = Java , frame = trBL , escapeinside={(*@}{@*)}]{../chapter_04/src/chapter_04/Referenzen.java}
\paragraph{Punkt.java}\
\lstinputlisting[language = Java , frame = trBL , escapeinside={(*@}{@*)}]{../chapter_04/src/chapter_04/Punkt.java}

\subsubsection{Testdokumentation}

\subsubsection{Benutzungshinweise}
Navigieren Sie in der Kommandozeile zum dem Ordner, wo sich die Java Datei befindet.
Danach führen sie "javac Referenzen.java\dq \space auf. Jetzt können Sie das Programm mit
"java Referenzen\dq \space starten.

\subsubsection{Anwendungsbeispiel}
Nach dem Aufruf von java Referenzen, sollten wir nun folgendes sehen:
\begin{lstlisting}[frame = trBL , escapeinside={(*@}{@*)}]
[sebastian@laptop bin]$ java Referenzen 
Ist ungleich
Ist ungleich
Ist gleich
[sebastian@laptop bin]$  
\end{lstlisting}
